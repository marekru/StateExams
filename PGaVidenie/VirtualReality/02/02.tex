\documentclass[]{article}

%opening
\title{Prednaska 2}
\author{}

\begin{document}

\maketitle


\section{Organizacia}

Po $11^{15}$ C Obhajoba Dizertacnej prace \textit{Z.Berger-Haladovej} \textbf{Argumented Reality} \\ \\
SNG $\rightarrow$ \textbf{QUALCOMM} \\
Knizka zo zvolena o mdoelovani lesa ! \\

\section{}
\begin{flushleft}
Kukame Mnemonic - headmounted display, datova rukavica \\

\textit{Ivan Sutterland} - Posledny/Definite display (vizia)\\

\textit{Myron Krueger} - VideoPlace(2D) - 1. system VR:{
	Pouzivatel snimany kamerou a premietany do obrazoveho(virtualneho) priestoru
	Nazival VR - Artificial Reality
}\\

\textit{Jaron Lanier} - specatil pojem VR, texhnicky spravil HeadMounted Display, DataGlove\\

DataSuit - vyzia \\

Multi-User Divided? Virtual Reality \\

Virtualna Jaskyna v ZV - simulacia rastu lesa \\
Virutalne jaskyne - hlavne armadne ucely \\

2001 \textbf{Ferwerda} \\
Physical realism \\
	- presne hodnoty jasu, materialy, geometria ... fyzikalne korektna \\
Photorealism \\
	- nerozlisitelne fotka a vygenerovany obrazok
	Lumigraph [Gortler]
	Image Based Lightning [Debevec]
	Realistickejsie - rozmazat, zasumiet \\
Functional realism \\
	- nakresy napr. \\

	


\end{flushleft}



\end{document}
